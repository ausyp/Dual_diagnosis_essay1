
\documentclass[a4paper,man,british]{apa6}
\usepackage[british]{babel}
\usepackage[utf8]{inputenc}
\usepackage{csquotes}
\usepackage[hidelinks]{hyperref}
\usepackage[style=apa]{biblatex}
\DeclareLanguageMapping{british}{british-apa}

% maps apacite commands to biblatex commands
\let \citeNP \cite
\let \citeA \textcite
\let \cite \parencite

\addbibresource{zotero_references.bib}
\addbibresource{bibliography.bib}

\title{Description and critique of Pill Testing Campaign}
\shorttitle{Pill testing campaigns}
\author{Austin Paul}
\affiliation{RMIT UNIVERSITY \\ s3634517 \\ Word Count : 986}




\begin{document}

\maketitle

\section{Introduction}

"Pill Testing Saves lives" is the undeniable claim that makes up the core of the argument for Pill Testing. Pill testing is also referred to as Drug checking by some media outlets. The idea of spending tax payer money to make illegal drugs safer, is a pill too hard to swallow for many and political leaders across the country has voiced their disapproval like NSW Premier Gladys Berejiklian  \cite{visentin_pill_2018,james_tas_2019} and their approval like 	Greens Leader Richard Di Natale \cite{patricia_greens_2019}. "War has no winners, only widows" is a quote from my all time favourite movie Arrival, which can be very meaningfully attributed to the war on drugs \cite{arrival2015} as it also left many widows as it threaded through two generations \cite{wodak_failure_2015,}. The reason I chose this campaign for my essay is that it is one of the most controversial health campaign in contemporary Australia and I think that pill testing provides an opportunity to make an informed decision to the user and may inspire change rather than impose it.

\section{History}

Australia's first drug law has its roots embedded in a lucrative colonial business venture where opium produced in British India was smuggled into China to fund an increasingly expensive empire and its trade with china \cite{keller_chinas_2011}. As migrant workers from china came to Australia during the gold rush of 1850s there was coinciding increase in recreational opium use in Australia, as opium smuggled in by the British was widely used in china for recreational purpose by that time. Unsurprisingly, Queensland introduced Australia's first law to limit the supply of narcotics in 1897, this came not as a moral position but as a direct response to an anti-Chinese sentiment in the Australian political milieu at that time  \cite{grant_functional_2014}.

\newpage


\section{Description of the campaign}

Pill testing is a reflection global transition towards a harm minimisation strategy from a  abstention and criminalisation strategy. The generation long War drugs has to a great extend failed to achieve its goal and can be considered a further failure if the cost of the campaign is taken into account \cite{wodak_failure_2015}. 
\newline
Pill testing campaign's are primarily targeted at young Australian party, festival and nightlife attendees\cite{barratt_pill_2018}. 
\newline 
The program involves a team of professionals and technicians who uses an analytical instrument like Fourier transform infrared (FTIR) spectrometer to determine the chemical composition of the drug brought in by a person about to consume it. Before the test the person bringing in the drug (client) is asked about their expectation of the contents of the drug. Once results are obtained from the checking instrument it is compared with the expectations and harm reduction advice are provided in accordance with the results and are also advised about the potential adverse outcomes. \cite{tupper_initial_2018,makkai2018report}.
\newline
As a harm reduction strategy Pill testing is in line with the logic of needle exchange programes and safe injecting rooms which in no small measure  played a significant role in reducing the incidence of HIV in intravenous drug users. But unlike the needle exchange programmes, pill testing tries to address the concerns arising from uncertainty in the contents of illicit drugs as their manufacturers have zero accountability, consistency or quality control \cite{winstock_ecstasy_2001}.
\newline
 Although Australia's  first government-approved pill testing trial was implemented on 29 April 2018 at the Groovin’
The Moo (GTM) festival in Canberra by Pill Testing Australia, the final results are not out yet, But early reports and progress report suggest that the trial identified the same inconsistent manufacturer quality and adulteration with multiple other substances and in some instances it identified a varying spectrum of quality of the same drug in the sample further exemplifying lack of manufacturer quality control \cite{anna_pill_2019,makkai2018report}.

\newpage

\section{Criticism and effectiveness of the Campaign}

One of the most common criticism of Pill testing is that it gives the user a false sense of security \cite{winstock_ecstasy_2001,anna_pill_2019}. Although all available research does in fact express some degree of concern with the accuracy of on site testing instruments, It is also worth mentioning that the instruments does indicate that it is unable determine the contents of the sample and an Australian study showed 76\% of frequent ecstasy consumers stated that would not take a pill if a test can not determine its contents \cite{barratt_pill_2018}\parencite[as cited in][]{johnston2006survey}.
\newline
Another criticism of pill testing is that it encourages young people to take drugs, or to take more drugs than they would if such services were not available but evidence suggest that not to be the case as there is no available research shows an increase in drug use after the introduction of pill testing \cite{brunt_drug_2017}. Even though this  argument can be seen as an "argumentum ad ignorantiam" it is still valid to show that contrary can also not be true.
\newline
Another argument against pill testing is that Drugs mainly MDMA in itself is dangerous and can cause fatalities and that there is no safe amount to use \cite{kirkham_no_2018}. Even though it is true that MDMA at higher doses are toxic and potentially lethal, but there are clinically approved studies that show that MDMA assisted psycho therapy helps in PTSD \cite{chabrol_mdma_2013}.
\newline
Critics of drug checking often points out that on site testing can not detect newer drugs \cite{leibie_pill_2017}. But it is also true that when a New drug named N-BOMB first caused several deaths in Melbourne \cite{woods_they_2017},It was Spanish Drug checking NGO that was able to determine what the drug was as they received a sample from an anonymous source \cite{didier_our_2017}.

\newpage

\section{Conclusion}

In this essay I have attempted lay down an argument to support pill testing and present a reasonably unbiased discussion of the Campaign, but the reality of the matter is that pill testing comes with its own set of problems; some technological, some socio-cultural and some stems from a lack of scientific understanding of drug interaction's. It is not to say that pill testing won't work but there is also very less evidence to show that it does at this moment.In order for this limbo to changes we need to develop better equipment's to identify substances faster, obtain more funding to conduct trails with more sample size for longer periods by taking more relevant factors into account.

\printbibliography

\end{document}
