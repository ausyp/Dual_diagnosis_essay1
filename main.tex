
\documentclass[a4paper,man,british]{apa6}
\usepackage[british]{babel}
\usepackage[utf8]{inputenc}
\usepackage{csquotes}
\usepackage[hidelinks]{hyperref}
\usepackage[style=apa]{biblatex}
\DeclareLanguageMapping{british}{british-apa}

% maps apacite commands to biblatex commands
\let \citeNP \cite
\let \citeA \textcite
\let \cite \parencite

\addbibresource{zotero_references.bib}
\addbibresource{bibliography.bib}

\title{Description and critique of a health promotion campaign}
\shorttitle{health promotion campaigns}
\author{Austin Paul}
\affiliation{RMIT UNIVERSITY \\ s3634517 \\ Word Count : 1000}




\begin{document}

\maketitle

\section{Introduction}

"Pill Testing Saves lives" is the undeniable claim that makes up the core of the argument for Pill Testing. Pill testing is also referred to as Drug checking by some media outlets. The idea of spending tax payer money to make illegal drugs safer, is a pill too hard to swallow for many and political leaders across the country has voiced their disapproval like NSW Premier Gladys Berejiklian  \cite{visentin_pill_2018,james_tas_2019} and their approval like 	Greens Leader Richard Di Natale \cite{patricia_greens_2019}. "War has no winners, only widows" is a quote from my all time favourite movie Arrival, which can be very meaningfully attributed to the war on drugs \cite{arrival2015} as it also left many widows as it threaded through two generations \cite{wodak_failure_2015,}. The reason I chose this campaign for my essay is that it is one of the most controversial health campaign in contemporary Australia and I think that pill testing provides an opportunity to make an informed decision to the user and may inspire change rather than impose it.

\section{History}

Australia's first drug law has its roots embedded in a lucrative colonial business venture where opium produced in British India was smuggled into China to fund an increasingly expensive empire and its trade with china \cite{keller_chinas_2011}. As migrant workers from china came to Australia during the gold rush of 1850s there was coinciding increase in recreational opium use in Australia, as opium smuggled in by the British was widely used in china for recreational purpose by that time. \cite{grant_functional_2014}. 

\newpage

\section{Conclusion}

vgtffth

\printbibliography

\end{document}
