\documentclass[a4paper,man,british]{apa6}
\usepackage[british]{babel}
\usepackage[utf8]{inputenc}
\usepackage{csquotes}
\usepackage[hidelinks]{hyperref}
\usepackage[style=apa]{biblatex}
\DeclareLanguageMapping{british}{british-apa}

% maps apacite commands to biblatex commands
\let \citeNP \cite
\let \citeA \textcite
\let \cite \parencite

\addbibresource{zotero_references.bib}
\addbibresource{bibliography.bib}

\title{Description and critique of a health promotion campaign}
\shorttitle{health promotion campaigns}
\author{Austin Paul}
\affiliation{RMIT UNIVERSITY \\ s3634517}



\begin{document}

\maketitle

\section{Introduction}

"Pill testing Save lives" is the undeniable statement that makes up the core of the argument for Pill Testing.  War has no winners only widows is a quote from my favourite movie which can be, very meaningfully attributed to the war on drugs \cite{arrival2015}. Pill testing is also referred to as Drug checking by some media outlets. The reason I chose this campaign for my essay is that it is one of the most controversial health campaign in contemporary Australia. On
\section{History}

\newpage
\section{Conclusion}

vgtffth

\printbibliography

\end{document}
